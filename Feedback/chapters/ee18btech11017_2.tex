\begin{enumerate}[label=\thesection.\arabic*.,ref=\thesection.\theenumi]
\numberwithin{equation}{enumi}

\item A two-pole amplifier for which $G_{o}=10^{3}$ and having
poles at 1 MHz and 10 MHz is to be connected as a differentiator. On the basis of the rate-of-closure rule, what is the smallest differentiator time constant for which operation is stable? What are the corresponding gain and phase
margins? \\
\solution The differentiator circuit is as shown in Fig. \ref{fig:ee18btech11017_differentiator}.Here $\tau=RC$
\renewcommand{\thefigure}{\theenumi.\arabic{figure}}
%
\begin{figure}[!ht]
	\begin{center}
		
		\resizebox{\columnwidth}{!}{\begin{circuitikz}
\ctikzset{bipoles/length=1cm}

\draw 
(0, 0) node[op amp] (opamp) {}
(opamp.-) -- (-2.5,0.35) to[C,l_=$C$] (-3.2, 0.35) to (-3.8, 0.35) to (-4.2, 0.35) node at(-4.5,0.35){$V_i$}
(opamp.-)--(-1.5,0.35) --(-1.5,1.25) to[R=$R$] (1,1.25) -- (1,0) --(2,0) node at(2.3,0){$V_o$}
(opamp.out) to (1.5,0)--(2,0) 
(opamp.+) -- (-0.6,-0.35) --(-1.5,-0.35) node[ground]{}

node at(-1.5,0.001,-0.2){$V_f$}
;\end{circuitikz}



}
	\end{center}
\caption{}
\label{fig:ee18btech11017_differentiator}
\end{figure}
%

\renewcommand{\thefigure}{\theenumi}

For the above circuit using supersposition theorem we can write,
\begin{align}
V_{f}=\brak{\frac{R}{R+\frac{1}{SC}}}V_{i}+\brak{\frac{\frac{1}{SC}}{R+\frac{1}{SC}}}V_{o}
\label{eq:ee18btech11017_fb1}
\end{align}

\item Draw the equivalent control system. \\
\solution 
\begin{figure}[!ht]
	\begin{center}
		
		\resizebox{\columnwidth}{!}{\tikzstyle{input} = [coordinate]
\tikzstyle{output} = [coordinate]
\tikzstyle{block} = [draw, rectangle]
\tikzstyle{sum} = [draw, circle]

\begin{tikzpicture}[auto, node distance=2cm,>=latex']
    \node [input, name=input] {$I_s$};
    \node [block, right of=input] (gain) {$K$};
    \node [sum, right of=gain] (sum) {};
    \node [block, right of=sum] (controller) {$G$};
    \node [output, right of=controller] (output) {};
    \node [block, below of=controller] (feedback) {$H$};
    \draw [draw,->] (input) -- node {$V_{i}$} (gain);
    \draw [->] (gain) -- node {} (sum);
    \draw [->] (sum) -- node  {$V_f$}(controller);
    \draw [->] (controller) --node[name=y]{$V_o$}(output);
    \draw [->] (y) |- (feedback);
    \draw [->] (feedback) -| node[pos=0.95]{$-$}  node [near end] {} (sum);
    \draw [->] (feedback) -| node[pos=1.15]{$+$}  node [near end] {} (sum);
\end{tikzpicture}
}
	\end{center}
\caption{}
\label{fig:ee18btech11017_blockdiagram}
\end{figure}
%

\renewcommand{\thefigure}{\theenumi}
\begin{table}[!ht]
\centering


%%  This section checks if we are begin input into another file or  %%
%%  the file will be compiled alone. First use a macro taken from   %%
%%  the TeXbook ex 7.7 (suggestion of Han-Wen Nienhuys).            %%
\def\ifundefined#1{\expandafter\ifx\csname#1\endcsname\relax}


%%  Check for the \def token for inputed files. If it is not        %%
%%  defined, the file will be processed as a standalone and the     %%
%%  preamble will be used.                                          %%
\ifundefined{inputGnumericTable}

%%  We must be able to close or not the document at the end.        %%
	\def\gnumericTableEnd{\end{document}}


%%%%%%%%%%%%%%%%%%%%%%%%%%%%%%%%%%%%%%%%%%%%%%%%%%%%%%%%%%%%%%%%%%%%%%
%%                                                                  %%
%%  This is the PREAMBLE. Change these values to get the right      %%
%%  paper size and other niceties.                                  %%
%%                                                                  %%
%%%%%%%%%%%%%%%%%%%%%%%%%%%%%%%%%%%%%%%%%%%%%%%%%%%%%%%%%%%%%%%%%%%%%%

	\documentclass[12pt%
			  %,landscape%
                    ]{report}
       \usepackage[latin1]{inputenc}
       \usepackage{fullpage}
       \usepackage{color}
       \usepackage{array}
       \usepackage{longtable}
       \usepackage{calc}
       \usepackage{multirow}
       \usepackage{hhline}
       \usepackage{ifthen}
%%  End of the preamble for the standalone. The next section is for %%
%%  documents which are included into other LaTeX2e files.          %%
\else

%%  We are not a stand alone document. For a regular table, we will %%
%%  have no preamble and only define the closing to mean nothing.   %%
    \def\gnumericTableEnd{}

%%  If we want landscape mode in an embedded document, comment out  %%
%%  the line above and uncomment the two below. The table will      %%
%%  begin on a new page and run in landscape mode.                  %%
%       \def\gnumericTableEnd{\end{landscape}}
%       \begin{landscape}


%%  End of the else clause for this file being \input.              %%
\fi

%%%%%%%%%%%%%%%%%%%%%%%%%%%%%%%%%%%%%%%%%%%%%%%%%%%%%%%%%%%%%%%%%%%%%%
%%                                                                  %%
%%  The rest is the gnumeric table, except for the closing          %%
%%  statement. Changes below will alter the table's appearance.     %%
%%                                                                  %%
%%%%%%%%%%%%%%%%%%%%%%%%%%%%%%%%%%%%%%%%%%%%%%%%%%%%%%%%%%%%%%%%%%%%%%

\providecommand{\gnumericmathit}[1]{#1} 
%%  Uncomment the next line if you would like your numbers to be in %%
%%  italics if they are italizised in the gnumeric table.           %%
%\renewcommand{\gnumericmathit}[1]{\mathit{#1}}
\providecommand{\gnumericPB}[1]%
{\let\gnumericTemp=\\#1\let\\=\gnumericTemp\hspace{0pt}}
 \ifundefined{gnumericTableWidthDefined}
        \newlength{\gnumericTableWidth}
        \newlength{\gnumericTableWidthComplete}
        \newlength{\gnumericMultiRowLength}
        \global\def\gnumericTableWidthDefined{}
 \fi
%% The following setting protects this code from babel shorthands.  %%
 \ifthenelse{\isundefined{\languageshorthands}}{}{\languageshorthands{english}}
%%  The default table format retains the relative column widths of  %%
%%  gnumeric. They can easily be changed to c, r or l. In that case %%
%%  you may want to comment out the next line and uncomment the one %%
%%  thereafter                                                      %%
\providecommand\gnumbox{\makebox[0pt]}
%%\providecommand\gnumbox[1][]{\makebox}

%% to adjust positions in multirow situations                       %%
\setlength{\bigstrutjot}{\jot}
\setlength{\extrarowheight}{\doublerulesep}

%%  The \setlongtables command keeps column widths the same across  %%
%%  pages. Simply comment out next line for varying column widths.  %%
\setlongtables

\setlength\gnumericTableWidth{%
	50pt+%
	50pt+%
	50pt+%
0pt}
\def\gumericNumCols{3}
\setlength\gnumericTableWidthComplete{\gnumericTableWidth+%
         \tabcolsep*\gumericNumCols*2+\arrayrulewidth*\gumericNumCols}
\ifthenelse{\lengthtest{\gnumericTableWidthComplete > \linewidth}}%
         {\def\gnumericScale{\ratio{\linewidth-%
                        \tabcolsep*\gumericNumCols*2-%
                        \arrayrulewidth*\gumericNumCols}%
{\gnumericTableWidth}}}%
{\def\gnumericScale{1}}

%%%%%%%%%%%%%%%%%%%%%%%%%%%%%%%%%%%%%%%%%%%%%%%%%%%%%%%%%%%%%%%%%%%%%%
%%                                                                  %%
%% The following are the widths of the various columns. We are      %%
%% defining them here because then they are easier to change.       %%
%% Depending on the cell formats we may use them more than once.    %%
%%                                                                  %%
%%%%%%%%%%%%%%%%%%%%%%%%%%%%%%%%%%%%%%%%%%%%%%%%%%%%%%%%%%%%%%%%%%%%%%

\ifthenelse{\isundefined{\gnumericColA}}{\newlength{\gnumericColA}}{}\settowidth{\gnumericColA}{\begin{tabular}{@{}p{50pt*\gnumericScale}@{}}x\end{tabular}}
\ifthenelse{\isundefined{\gnumericColB}}{\newlength{\gnumericColB}}{}\settowidth{\gnumericColB}{\begin{tabular}{@{}p{60pt*\gnumericScale}@{}}x\end{tabular}}
\ifthenelse{\isundefined{\gnumericColC}}{\newlength{\gnumericColC}}{}\settowidth{\gnumericColC}{\begin{tabular}{@{}p{60pt*\gnumericScale}@{}}x\end{tabular}}

\begin{tabular}[c]{%
	b{\gnumericColA}%
	b{\gnumericColB}%
	b{\gnumericColC}%
	}

%%%%%%%%%%%%%%%%%%%%%%%%%%%%%%%%%%%%%%%%%%%%%%%%%%%%%%%%%%%%%%%%%%%%%%
%%  The longtable options. (Caption, headers... see Goosens, p.124) %%
%	\caption{The Table Caption.}             \\	%
% \hline	% Across the top of the table.
%%  The rest of these options are table rows which are placed on    %%
%%  the first, last or every page. Use \multicolumn if you want.    %%

%%  Header for the first page.                                      %%
%	\multicolumn{3}{c}{The First Header} \\ \hline 
%	\multicolumn{1}{c}{colTag}	%Column 1
%	&\multicolumn{1}{c}{colTag}	%Column 2
%	&\multicolumn{1}{c}{colTag}	\\ \hline %Last column
%	\endfirsthead

%%  The running header definition.                                  %%
%	\hline
%	\multicolumn{3}{l}{\ldots\small\slshape continued} \\ \hline
%	\multicolumn{1}{c}{colTag}	%Column 1
%	&\multicolumn{1}{c}{colTag}	%Column 2
%	&\multicolumn{1}{c}{colTag}	\\ \hline %Last column
%	\endhead

%%  The running footer definition.                                  %%
%	\hline
%	\multicolumn{3}{r}{\small\slshape continued\ldots} \\
%	\endfoot

%%  The ending footer definition.                                   %%
%	\multicolumn{3}{c}{That's all folks} \\ \hline 
%	\endlastfoot
%%%%%%%%%%%%%%%%%%%%%%%%%%%%%%%%%%%%%%%%%%%%%%%%%%%%%%%%%%%%%%%%%%%%%%

\hhline{|-|-|-}
	 \multicolumn{1}{|p{\gnumericColA}|}%
	{\gnumericPB{\centering}\textbf{Parameters}}
	&\multicolumn{1}{p{\gnumericColB}|}%
	{\gnumericPB{\centering}\textbf{Definition}}
	&\multicolumn{1}{p{\gnumericColC}|}%
	{\gnumericPB{\centering}\textbf{For given circuit}}

	
\\


\hhline{|---|}
	 \multicolumn{1}{|p{\gnumericColA}|}%
	{\gnumericPB{\centering}{Open loop gain}}
	&\multicolumn{1}{p{\gnumericColB}|}%
	{\gnumericPB{\centering}G}
	&\multicolumn{1}{p{\gnumericColC}|}%
	{\gnumericPB{\centering}G}

\\
\hhline{|---|}
	 \multicolumn{1}{|p{\gnumericColA}|}%
	{\gnumericPB{\centering}Feedback factor}
	&\multicolumn{1}{p{\gnumericColB}|}%
	{\gnumericPB{\centering}H}
	&\multicolumn{1}{p{\gnumericColC}|}%
	{\gnumericPB{\centering}{$\frac{1}{1+sRC}$}}

\\
\hhline{|---|}
	 \multicolumn{1}{|p{\gnumericColA}|}%
	{\gnumericPB{\centering}Loop gain}
	&\multicolumn{1}{p{\gnumericColB}|}%
	{\gnumericPB{\centering}GH}
	&\multicolumn{1}{p{\gnumericColC}|}%
	{\gnumericPB{\centering}{$\frac{G}{1+sRC}$}}

\\


\hhline{|-|-|-|}
    \multicolumn{1}{|p{\gnumericColA}|}%
	{\gnumericPB{\centering}Gain Factor}
	&\multicolumn{1}{p{\gnumericColB}|}%
	{\gnumericPB{\centering}K}
	&\multicolumn{1}{p{\gnumericColC}|}%
	{\gnumericPB{\centering}{$\frac{-sRC}{1+sRC}$}}

\\
\hhline{|---|}
	 \multicolumn{1}{|p{\gnumericColA}|}%
	{\gnumericPB{\centering}Closed loop gain }
	&\multicolumn{1}{p{\gnumericColB}|}%
	{\gnumericPB{\centering}{$K\frac{G}{1+GH}$}}
	&\multicolumn{1}{p{\gnumericColC}|}%
	{\gnumericPB{\centering}{$\frac{G(-sRC)}{1+G+sRC}$}}

\\
\hhline{|-|-|-|}
\end{tabular}

\ifthenelse{\isundefined{\languageshorthands}}{}{\languageshorthands{\languagename}}
\gnumericTableEnd

\caption{}
\label{table:ee18btech11017_table1}
\end{table}


\item Find feedback factor H and open loop gain G. \\
\solution
From the above table,
\begin{align}
H(s) =\frac{1}{1+sRC} \\
\implies \frac{1}{H(s)}=1+sRC \approx sRC \\
G(s) =\frac{10^3}{\brak{1+\frac{s}{2\pi \cdot 10^6}}\brak{1+\frac{s}{2\pi \cdot 10^7}}}
\end{align}

\item Rate of Closure rule \\
\solution The rule states that at intersection of $20\log\abs{\frac{1}{H(\j\omega)}}$ and $20\log\abs{G(\j\omega)}$ the difference of the slopes (called the rate of closure) should not exceed 20dB/decade.

The Bode plot for $\frac{1}{\abs{H\brak{\j\omega}}}$ has a slope of +20dB/decade.Therefore , the closed-loop amplifier will be stable if $20\log\brak{\frac{1}{H}}$ line intersects the $20\log\abs{G}$ curve at a point on the 0dB/decade segment.\\
The 0dB/decade segment of the $20\log\abs{G}$ curve is $G_{o}=10^{3}$.We should test the stability at the pole frequency $\omega_{P1}=2\pi\cdot10^{6}$ which seperate the 0dB/decade segment and the -20dB/decade.Thus we can write that as,

\begin{align}
20\log\abs{G_{o}}-20\log\brak{\frac{1}{H(\j\omega_{P1})}} &\leq 20dB \\
\implies \abs{G_{o}H(\j\omega_{P1})} & \leq 1
\end{align}
Substituting,
\begin{align}
G_{o} \frac{1}{\omega_{P1} \tau} \leq 1 \\
\therefore \tau \geq \frac{G_{o}}{\omega_{P1}}= \frac{10^{3}}{2\pi\cdot 10^{6}}=159.15 \mu s \\
\label{eq:ee18btech11017_fb2}
\implies \tau_{min}=159.15 \mu s
\end{align}


\item
Find Phase Margin and Gain Margin for $\tau=\tau_{min}$. \\
\solution 
From the Equation \eqref{eq:ee18btech11017_fb2} 
\begin{align}
\text{ At} , \tau =\tau_{min} \\
\abs{G_{o}H(\j\omega_{P1})} =1 \\
\implies P.M = \angle G(\j\omega_{P1})H(\j\omega_{P1})+180\degree
\end{align}

So,
\begin{align}
\angle G\brak{\j\omega_{P1}} &= -tan^{-1}\brak{\frac{\omega_{P1}}{2\pi\cdot10^{6}}}-tan^{-1}\brak{\frac{\omega_{P1}}{2\pi\cdot10^{7}}} \\
&=-50.71\degree \\
\angle H\brak{\j\omega_{P1}} &=-90\degree \\
\implies P.M &= 180\degree-50.71\degree-90\degree =39.29\degree \\
\therefore P.M &= 39.29\degree
\end{align}

For Gain Margin we need to find $\omega_{180}$ such that 
\begin{align}
\angle G(\j\omega_{180})H(\j\omega_{180})= -180\degree \\
\implies -tan^{-1}\brak{\frac{\omega_{180}}{2\pi\cdot10^{6}}}-tan^{-1}\brak{\frac{\omega_{180}}{2\pi\cdot10^{7}}}-90\degree &=-180\degree
\end{align}
Solving we get,
\begin{align}
\omega_{180}&=19.98M rad/s
\end{align}
The Gain Margin is as follows,
\begin{align}
GM &= -20\log\abs{G\brak{\j\omega_{180}}H\brak{\j\omega_{180}}} \\
&= -20\log\abs{G\brak{\j\omega_{180}}}+20\log\abs{\frac{1}{H\brak{\j\omega_{180}}}} \\
&= -49.12+70.05 \\
\therefore G.M &= 20.93 dB
\end{align}




\end{enumerate}
